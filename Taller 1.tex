\documentclass[oneside]{article}
% Título y autor(es):
\title{Resumen de Taller de programación I (75.42)}
\author{Lafroce Matías}

\usepackage[spanish]{babel}
\usepackage{amsmath,bm,times}
\usepackage[a4paper,headheight=16pt,scale={0.7,0.8},hoffset=0.5cm]{geometry}
\usepackage{pdfpages}
\usepackage[utf8]{inputenc}
\usepackage{lastpage}
\usepackage{float}
\usepackage{array}
\usepackage{listings}
\usepackage{anysize}
\usepackage{pdfpages}
\numberwithin{equation}{section}
\numberwithin{figure}{section}
\numberwithin{table}{section}
\usepackage{fancyhdr}
\usepackage[hang,bf]{caption}
\usepackage{graphicx}
\usepackage{svg}
\usepackage{pgf,tikz}
\usepackage{tikz-timing}
\usepackage{setspace}
\usepackage{verbatim}
\usepackage{booktabs}

\includepdfset{pagecommand=\thispagestyle{plain}}
\usetikzlibrary{shapes,arrows,positioning,shadows,trees,automata}

\pdfcompresslevel=9
\newcommand{\imgdir}{includes}
\graphicspath{{\imgdir/}}

\everymath{\displaystyle}
\newcommand{\vect}[1]{\overline{\textbf{#1}}}
\newcommand{\bayes}{\mathop{\lessgtr}}

\newcommand{\fil}[1]{\text{Fil}(#1)}
\newcommand{\col}[1]{\text{Col}(#1)}
\newcommand{\nul}[1]{\text{Nul}(#1)}
\newcommand{\im}[1]{\text{Im}(#1)}
\newcommand{\rg}[1]{\text{rango}\left(#1\right)}
\newcommand{\dime}[1]{\text{Dim}(#1)}
\newcommand{\dete}[1]{\left\vert #1 \right\vert}
\newcommand{\tr}[1]{\text{tr}(#1)}
\newcommand{\gen}[1]{\text{gen}\left\{#1\right\}}
\newcommand{\produ}[1]{(#1)}

\tikzstyle{parametros} = [draw=red, fill=gray!10, very thick,rectangle, rounded corners, inner sep=10pt, inner ysep=20pt]
\tikzstyle{fancytitle} =[fill=red, text=white]

\definecolor{darkgreen}{rgb}{0,0.5,0}

\lstset{          % Elijo formato de bloques de código fuente
	backgroundcolor=\color{white},
	basicstyle=\ttfamily\footnotesize,
	commentstyle=\color{darkgreen},
	language=c
}

%------------------------- Inicio del documento ---------------------------

\begin{document}
% Hago que en la cabecera de página se muestre a la derecha la sección, y en el pie, en número de página a la derecha:
\pagestyle{fancy}
\renewcommand{\sectionmark}[1]{\markboth{}{\thesection\ \ #1}}
\lhead{}
\chead{}
\rhead{\rightmark}
\lfoot{Taller de programación I (75.42)}
\cfoot{}
\rfoot{P\'agina \thepage\ de \pageref{LastPage}}
% Carátula:
\begin{titlepage}

\thispagestyle{empty}

\begin{center}
\includegraphics[scale=0.3]{fiuba}\\
\large{\textsc{Universidad de Buenos Aires}}\\
\large{\textsc{Facultad De Ingeniería}}\\
\small{A\~no 2015 - 1\textsuperscript{er} Cuatrimestre}
\end{center}

\begin{center}
\Large{\underline{\textsc{Taller de programación I (75.42)}}}

\vspace*{2cm}

\textbf{\begin{LARGE}
Resumen de Taller de programación I\\
\end{LARGE}}
\end{center}

\vspace*{3cm}

\begin{tabbing}
\hspace{2cm}\=\+
\\
	AUTOR:\hspace{-1cm}\=\+\hspace{1cm}\=\hspace{6cm}\=\\
		Lafroce Matías 	\>\>- \ 91378\\
		$\langle$mlafroce@gmail.com$\rangle$\\
		\\
\end{tabbing}
\end{titlepage}
% Hago que las páginas se comiencen a contar a partir de aquí:
\setcounter{page}{1}
% Pongo el índice en una página aparte:
\tableofcontents
%\listoffigures
%\listoftables
\newpage
% Inicio del TP:
\marginsize{1cm}{1cm}{1cm}{1cm}
%\setcounter{chapter}{1}

	\section{Sockets}
		\subsection{int socket(int domain, int type, int protocol)}
		Crea un nuevo socket y devuelve su número de \textit{socket descriptor} (o -1 si hay un error).\\
		\begin{tikzpicture}
			\node [parametros] (box){%
				\begin{minipage}{0.80\textwidth}
					\begin{itemize}
						\item \textbf{domain}: Define si la familia de protocolos de la conexión. Algunos valores usados son: PF\_LOCAL (comunicación local), PF\_INET (IPv4), PF\_INET6 (IPv6).
						\item \textbf{type}: Define el tipo de conexión. Algunos de los valores valores más usados son SOCK\_STREAM y SOCK\_DGRAM para protocolos TCP y UDP respectivamente.
						\item \textbf{protocol}: Define el protocolo a utilizar, se lo puede dejar en 0 para que se elija el apropiado según el tipo de conexión.
					\end{itemize}
				\end{minipage}
				};
			\node[fancytitle, right=10pt] at (box.north west) {Parámetros};
			\node[fancytitle, rounded corners] at (box.south) {$\aleph$};						
		\end{tikzpicture}	
		
		\subsection{int bind(int sockfd, const struct sockaddr *addr, socklen\_t addrlen)}
		Asocia al socket a una dirección. Devuelve 0 en éxito o -1 en caso de error.\\
		\begin{tikzpicture}
			\node [parametros] (box){%
				\begin{minipage}{0.80\textwidth}
					\begin{itemize}
						\item \textbf{sockfd}: el número del socket descriptor al que se le quiere asociar la dirección.
						\item \textbf{addr}: dirección a la que se quiere asociar el socket.\\
						La estructura que se utiliza generalmente es la siguiente:\\
						\begin{lstlisting}
struct sockaddr_in {
	sa_family_t sin_family; /* address family: AF_INET */
	in_port_t sin_port;     /* puerto en formato de red (pasarle htons(port)) */
	in_addr sin_addr;       /*IP a la que se quiere asociar */
};
						\end{lstlisting}
						\item \textbf{addrlen}: Tamaño de la estructura: sizeof(struct sockaddr\_in);
						Si el socket se utilizará como cliente, no es necesario bindearlo antes de hacer un connect.
					\end{itemize}
				\end{minipage}
				};
			\node[fancytitle, right=10pt] at (box.north west) {Parámetros};
			\node[fancytitle, rounded corners] at (box.south) {$\aleph$};						
		\end{tikzpicture}	
\pagebreak
\end{document}
